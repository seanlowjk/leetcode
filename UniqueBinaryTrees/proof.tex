%This is my super simple Real Analysis Homework template

\documentclass{article}
\usepackage[utf8]{inputenc}
\usepackage[english]{babel}
\usepackage[]{amsthm} %lets us use \begin{proof}
\usepackage[]{amssymb} %gives us the character \varnothing

\title{Unique Binary Trees}
\author{Low Jun Kai, Sean}
\date\today
%This information doesn't actually show up on your document unless you use the maketitle command below

\begin{document}
\maketitle %This command prints the title based on information entered above

\section*{Preface}

Given an integer n, return the number of structurally unique BSTs (binary search trees) which has exactly n nodes of unique values from 1 to n

\section*{Theorem} 

Let the function $f(n)$ denote the function where given $n$ nodes, we get the \# of structurally unique BSTs. 

\[
    f(n) =
    \left \{
    \begin{array}{lr}
      0 & \textbf{for } n = 0 \lor n = 1 \\
      \sum^{n-1}_{i=0} f(i) \cdot f(n - i -1) & \textbf{for } n > 1 \\ 
    \end{array}
     \right \}
\]

\section*{Proof of Correctness}

\begin{enumerate}

\item To prove the two base cases are true is \textbf{trivial}

\begin{enumerate}

\item There is only 1 way to arrange a BST with 0 nodes

\item There is only 1 way to arrange a BST with 1 node

\end{enumerate}

\item Assuming $f(n)$ is true, we have to prove that $f(n+1)$ holds true as well for some $n > 1$

\item To prove the correctness of this is equivalent to \textbf{Proving the correctness of the closed formula for Catalan Numbers}. As a result, we will skip this proof for this exercise. 

\end{enumerate}

\end{document}